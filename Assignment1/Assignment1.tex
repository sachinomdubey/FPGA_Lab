\documentclass[journal,12pt,twocolumn]{IEEEtran}

\usepackage{setspace}
\usepackage{gensymb}
\singlespacing
\usepackage[cmex10]{amsmath}

\usepackage{amsthm}
\usepackage{karnaugh-map}
\usepackage{mathrsfs}
\usepackage{txfonts}
\usepackage{stfloats}
\usepackage{bm}
\usepackage{cite}
\usepackage{cases}
\usepackage{subfig}
\usepackage{float}
\usepackage{longtable}
\usepackage{multirow}

\usepackage{enumitem}
\usepackage{mathtools}
\usepackage{steinmetz}
\usepackage{tikz}
\usepackage{circuitikz}
\usepackage{verbatim}
\usepackage{tfrupee}
\usepackage[breaklinks=true]{hyperref}
\usepackage{graphicx}
\usepackage{tkz-euclide}

\usetikzlibrary{calc,math}
\usepackage{listings}
    \usepackage{color}                                            %%
    \usepackage{array}                                            %%
    \usepackage{longtable}                                        %%
    \usepackage{calc}                                             %%
    \usepackage{multirow}                                         %%
    \usepackage{hhline}                                           %%
    \usepackage{ifthen}                                           %%
    \usepackage{lscape}     
\usepackage{multicol}
\usepackage{chngcntr}

\DeclareMathOperator*{\Res}{Res}

\renewcommand\thesection{\arabic{section}}
\renewcommand\thesubsection{\thesection.\arabic{subsection}}
\renewcommand\thesubsubsection{\thesubsection.\arabic{subsubsection}}

\renewcommand\thesectiondis{\arabic{section}}
\renewcommand\thesubsectiondis{\thesectiondis.\arabic{subsection}}
\renewcommand\thesubsubsectiondis{\thesubsectiondis.\arabic{subsubsection}}


\hyphenation{op-tical net-works semi-conduc-tor}
\def\inputGnumericTable{}                                 %%

\lstset{
%language=C,
frame=single, 
breaklines=true,
columns=fullflexible
}
\begin{document}


\newtheorem{theorem}{Theorem}[section]
\newtheorem{problem}{Problem}
\newtheorem{proposition}{Proposition}[section]
\newtheorem{lemma}{Lemma}[section]
\newtheorem{corollary}[theorem]{Corollary}
\newtheorem{example}{Example}[section]
\newtheorem{definition}[problem]{Definition}

\newcommand{\BEQA}{\begin{eqnarray}}
\newcommand{\EEQA}{\end{eqnarray}}
\newcommand{\define}{\stackrel{\triangle}{=}}
\bibliographystyle{IEEEtran}
\raggedbottom

\providecommand{\mbf}{\mathbf}
\providecommand{\pr}[1]{\ensuremath{\Pr\left(#1\right)}}
\providecommand{\qfunc}[1]{\ensuremath{Q\left(#1\right)}}
\providecommand{\sbrak}[1]{\ensuremath{{}\left[#1\right]}}
\providecommand{\lsbrak}[1]{\ensuremath{{}\left[#1\right.}}
\providecommand{\rsbrak}[1]{\ensuremath{{}\left.#1\right]}}
\providecommand{\brak}[1]{\ensuremath{\left(#1\right)}}
\providecommand{\lbrak}[1]{\ensuremath{\left(#1\right.}}
\providecommand{\rbrak}[1]{\ensuremath{\left.#1\right)}}
\providecommand{\cbrak}[1]{\ensuremath{\left\{#1\right\}}}
\providecommand{\lcbrak}[1]{\ensuremath{\left\{#1\right.}}
\providecommand{\rcbrak}[1]{\ensuremath{\left.#1\right\}}}
\theoremstyle{remark}
\newtheorem{rem}{Remark}
\newcommand{\sgn}{\mathop{\mathrm{sgn}}}
\providecommand{\abs}[1]{\left\vert#1\right\vert}
\providecommand{\res}[1]{\Res\displaylimits_{#1}} 
\providecommand{\norm}[1]{\left\lVert#1\right\rVert}
%\providecommand{\norm}[1]{\lVert#1\rVert}
\providecommand{\mtx}[1]{\mathbf{#1}}
\providecommand{\mean}[1]{E\left[ #1 \right]}
\providecommand{\fourier}{\overset{\mathcal{F}}{ \rightleftharpoons}}
%\providecommand{\hilbert}{\overset{\mathcal{H}}{ \rightleftharpoons}}
\providecommand{\system}{\overset{\mathcal{H}}{ \longleftrightarrow}}
	%\newcommand{\solution}[2]{\textbf{Solution:}{#1}}
\newcommand{\solution}{\noindent \textbf{Solution: }}
\newcommand{\cosec}{\,\text{cosec}\,}
\providecommand{\dec}[2]{\ensuremath{\overset{#1}{\underset{#2}{\gtrless}}}}
\newcommand{\myvec}[1]{\ensuremath{\begin{pmatrix}#1\end{pmatrix}}}
\newcommand{\mydet}[1]{\ensuremath{\begin{vmatrix}#1\end{vmatrix}}}
\numberwithin{equation}{subsection}

\makeatletter
\@addtoreset{figure}{problem}
\makeatother
\let\StandardTheFigure\thefigure
\let\vec\mathbf

\renewcommand{\thefigure}{\theproblem}

\def\putbox#1#2#3{\makebox[0in][l]{\makebox[#1][l]{}\raisebox{\baselineskip}[0in][0in]{\raisebox{#2}[0in][0in]{#3}}}}
     \def\rightbox#1{\makebox[0in][r]{#1}}
     \def\centbox#1{\makebox[0in]{#1}}
     \def\topbox#1{\raisebox{-\baselineskip}[0in][0in]{#1}}
     \def\midbox#1{\raisebox{-0.5\baselineskip}[0in][0in]{#1}}
\vspace{3cm}
\title{Assignment 1}
\author{Sachinkumar Dubey - EE20MTECH11009}
\maketitle
\newpage
\bigskip
\renewcommand{\thefigure}{\theenumi}
\renewcommand{\thetable}{\theenumi}
\noindent Download all codes from 
\begin{lstlisting}
https://github.com/sachinomdubey/FPGA_Lab/Assignment1/codes
\end{lstlisting}
%
and latex-tikz codes from 
%
\begin{lstlisting}
https://github.com/sachinomdubey/FPGA_Lab/Assignment1
\end{lstlisting}
\section{Problem}
\noindent(CBSE/CS/2019/6.c) Derive a Canonical POS expression for a Boolean function $F$, represented by the following TABLE \ref{TTF}:
\begin{table}[h!]
\centering
\begin{tabular}{|l|l|l|c|} 
\hline
$X$ & $Y$ & $Z$ & $F(X,Y,Z)$  \\ 
\hline
0 & 0 & 0 & 1         \\ 
\hline
0 & 0 & 1 & 0         \\ 
\hline
0 & 1 & 0 & 1         \\ 
\hline
0 & 1 & 1 & 0         \\ 
\hline
1 & 0 & 0 & 1         \\ 
\hline
1 & 0 & 1 & 1         \\ 
\hline
1 & 1 & 0 & 0         \\ 
\hline
1 & 1 & 1 & 0         \\
\hline
\end{tabular}
\caption{Truth table for Function F}
\label{TTF}
\vspace{-8mm}
\end{table}
\section{Solution}

\noindent Here, the output $F$ is '$0$' for four combinations of inputs. The corresponding Max terms are
$(X+Y+\bar{Z})$, $(X+\bar{Y}+\bar{Z})$, $(\bar{X}+\bar{Y}+Z)$, $(\bar{X}+\bar{Y}+\bar{Z})$. By doing logical AND of these four Max terms, we will get the Boolean function F,
\begin{align}
F = (X+Y+\bar{Z})\cdot(X+\bar{Y}+\bar{Z})\cdot \nonumber \\ 
(\bar{X}+\bar{Y}+Z)\cdot(\bar{X}+\bar{Y}+\bar{Z})
\end{align}
This is the canonical POS form for the Boolean function $F$. We can also represent this function in following two notations.
\begin{align}
F = M_1 \cdot M_3 \cdot M_6 \cdot M_7 \\
F = \prod M\left ( 1,3,5,7 \right )
\end{align}
\underline{Implementation using two input NAND gates}\\\\
Minimizing the function using K-maps,
\begin{figure}[H]
\centering
 \begin{karnaugh-map}[4][2][1][$B C$][$A$]
        \minterms{0,2,4,5}
        \maxterms{1,3,6,7}
        \implicant{1}{3}
        \implicant{7}{6}
    \end{karnaugh-map}
    \vspace{-6mm}
\end{figure}
\noindent We get the following minimized POS form,
\begin{align}
F &= (A+\bar{C})\cdot(\bar{A}+\bar{B})
\end{align}
Using Demorgan's law, we can write :
\begin{align}
F &=\overline{(\bar{A}\cdot C)}\cdot \overline{(A\cdot B)} \\
&= \overline{ \overline{ \overline{(\bar{A}\cdot C)}\cdot \overline{(A\cdot B)}}}
\end{align}
Implementing using two input NAND gate :
\begin{figure}[H]
\centering
\resizebox{\columnwidth}{!}
    {
    
\tikzset{every picture/.style={line width=0.75pt}} %set default line width to 0.75pt        

\begin{tikzpicture}[x=0.75pt,y=0.75pt,yscale=-1,xscale=1]

\draw   (335.1,58.89) -- (594.49,270.28) -- (334.1,663.59) -- (334.1,663.59) -- (73.72,270.28) -- cycle ;

\draw    (73.72,270.28) -- (594.49,270.28) ;

\draw    (335.1,58.89) -- (334.1,663.59) ;

\draw    (196.49,158.07) -- (209.49,176.07) ;

\draw    (465.49,152.07) -- (453.49,170.07) ;

\draw    (486.49,417.07) -- (502.61,431.11) ;

\draw    (481.49,425.07) -- (496.61,439.11) ;
 
\draw    (190.55,426.14) -- (174.55,440.14) ;

\draw    (195.55,434.14) -- (179.55,448.14) ;

\draw (43.08,247.59) node [anchor=north west][inner sep=0.75pt]   [align=left] {{\LARGE A}};

\draw (601.08,254.59) node [anchor=north west][inner sep=0.75pt]   [align=left] {{\LARGE B}};

\draw (326.08,25.59) node [anchor=north west][inner sep=0.75pt]   [align=left] {{\LARGE P}};

\draw (341.1,278.28) node [anchor=north west][inner sep=0.75pt]   [align=left] {{\LARGE M}};

\draw (345.08,657.36) node [anchor=north west][inner sep=0.75pt]   [align=left] {{\LARGE Q}};

\end{tikzpicture}

    }
\caption{Implementation using two input NAND gates}
\end{figure}
\end{document}